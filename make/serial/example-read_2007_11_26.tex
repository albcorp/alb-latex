%
% :Precis: Example readings for 26 NOV 2007
% :Authors: Andrew Burrow
% :Copyright: 2005-2008, 2013 Andrew Lincoln Burrow
%

%
%
%
% 26 NOV 2007
%

\chapter{26 NOV 2007}
\label{cha:read_2007_11_26:26-nov-2007}

Papers collected on agent systems for recommenders and team formation in
the setting of hypertext collaboration tools.

%
%
% Background
%

\section{Background}
\label{sec:read_2007_11_26:backgr}

%
% pan03:_light_ontol_repos

\begin{albResearchNote}{%
    \citet{pan03:_light_ontol_repos}%
  }{%
    \glossary{multi-agent~system}, \glossary{ontology~repository}%
  }
\item Describes the \emph{Otago Ontology Repository} and its RESTful
  architecture.  Ontologies are published as RDF documents via HTTP.
\end{albResearchNote}

%
% mui02:_notion_reput_multi_agent_system

\begin{albResearchNote}{%
    \citet{mui02:_notion_reput_multi_agent_system}%
  }{%
    \glossary{multi-agent~system}, \glossary{reputation}%
  }
\item Surveys existing work on reputation to construct a reputation
  typology.
\item Claims to quantitatively compare the different notions of
  reputation by simulation using evolutionary game theory.
\end{albResearchNote}

%
%
% Recommender systems
%

\section{Recommender Systems}
\label{sec:read_2007_11_26:recomm-syst}

One approach to supporting human group-formation via a multi-agent
system, is a recommender system implemented by personal information
agents.  In this scenario, each user is represented by a personal
information agent that make recommendations on collaborators and
documents reachable through the wiki.

%
% somlo03:_using_web_helper_agent_profil_query_gener

\begin{albResearchNote}{%
    \citet{somlo03:_using_web_helper_agent_profil_query_gener}%
  }{%
    \glossary{personal~information~agent},
    \glossary{recommender~system}%
  }
\item Describes a personal information agent called \emph{SurfAgent}
  that learns relevance to the user by positive feedback in order to
  recommend new documents.
\item Describes user profile in terms of \emph{TF-IDF vectors}, namely
  ranking by term frequency and inverse document frequency.
\item Distinguishes between \emph{retrieval queries} and \emph{filter
    queries}.
\end{albResearchNote}

%
% somlo04:_query

\begin{albResearchNote}{%
    \citet{somlo04:_query}%
  }{%
    \glossary{personal~information~agent},
    \glossary{recommender~system}%
  }
\item Describes a personal information agent called \emph{QueryTracker}
  that builds a user profile by ``unsupervised clustering methods on
  documents judged relevant by the user,'' in order to track persistent
  information needs by then constructing queries, and filtering the
  results.
\item Describes user profile in terms of \emph{TF-IDF vectors}, namely
  ranking by term frequency and inverse document frequency.
\item Finesses the output by performing change detection to cover the
  case where previous documents have changed in interesting ways as
  determined by the user profile.
\end{albResearchNote}

%
% birukov05:_implic

\begin{albResearchNote}{%
    \citet{birukov05:_implic}%
  }{%
    \glossary{multi-agent~system},
    \glossary{personal~information~agent},
    \glossary{recommender~system}%
  }
\item Describes a multi-agent system called \emph{Implicit} in which
  each personal information agent learns relevance to the user, and
  personal information agents cooperate to exchange information that
  improves the filtering query.
\item Claims that collaboration between personal information agents
  induces group behaviour in new users.
\item Note that the groups preexist and are implicitly considered
  reliable.
\end{albResearchNote}

%
% birukou06:_multi_agent_system_facil_scien_public_searc

\begin{albResearchNote}{%
    \citet{birukou06:_multi_agent_system_facil_scien_public_searc}%
  }{%
    \glossary{multi-agent~system},
    \glossary{personal~information~agent},
    \glossary{recommender~system}%
  }
\item Describes a multi-agent system that extends the capabilities and
  applicability of the system described in \citep{birukov05:_implic}.
\item Claims improvements to similar recommender system by accommodating
  groups composed of inexpert users, better filtering the
  recommendations, and exploiting implicit feedback found in the
  citation graph.
\item Claims expertise can be transmitted by cloning an expert's
  personal information agent.
\end{albResearchNote}

%
% yu03:_searc_social_networ

\begin{albResearchNote}{%
    \citet{yu03:_searc_social_networ}%
  }{%
    \glossary{multi-agent~system},
    \glossary{personal~information~agent}, \glossary{referral~system}%
  }
\item Describes a multi-agent system in which each personal information
  agent learns relevance to the user and maintains beliefs about the
  user's acquaintances, in order to perform distributed search through
  referrals in response to requests for information.
\item Describes personal information agents that model both expertise
  and sociability, and describes
\item Note the referral system could be construed as a recommender of
  people rather than documents.
\end{albResearchNote}

%
%
% Agent group-formation
%

\section{Agent Group-Formation}
\label{sec:read_2007_11_26:agent-group-form}

A useful analogy can be drawn between human group-formation and agent
group-formation in multi-agent systems.  Furthermore, this analogy can
be realised by noting that both collections of pages, and collections of
users are justifiable starting points for clustering users.  By setting
a multi-agent system the goal of forming groups of pages, it is possible
to influence human group-formation in the wiki.  Envisage a multi-agent
system in which agents develop beliefs about the clustering of pages,
and hence of users.

%
% legras03:_lotto

\begin{albResearchNote}{%
    \citet{legras03:_lotto}%
  }{%
    \glossary{multi-agent~system}, \glossary{agent~group-formation}%
  }
\item Describes a group-formation in a multi-agent system based on
  overhearing messages.
\item Extends earlier work to consider the case of many agents with
  limited memory.
\item Describes group-formation via partitions of the set of agents,
  where agents hold incomplete and possibly incompatible beliefs about
  the partition, and the agent system attempts to optimise the coherence
  of these distinct beliefs.
\item Describes small collection of propositional attitudes covering the
  required communications.
\end{albResearchNote}

%
% gaston05:_agent_organ_networ_dynam_team_format

\begin{albResearchNote}{%
    \citet{gaston05:_agent_organ_networ_dynam_team_format}%
  }{%
    \glossary{multi-agent~system}, \glossary{agent~group-formation}%
  }
\item Describes a multi-agent system in which agents make local
  modifications to the network by which agents locate one another.
\item Defines an \emph{agent-organized network} (AON) as ``an
  organizational network structure, or agent-to-agent interaction
  topology, that is the result of local rewiring decisions made by the
  individual agents.''
\item Considers AON strategies for cooperative multi-agent systems, and
  describes a strategy that preserves certain structural invariants on
  the network.
\item Demonstrates the affect of the AON strategy on dynamic
  group-formation in the agent system.
\end{albResearchNote}

%
%
% Human group-formation
%

\section{Human Group-Formation}
\label{sec:read_2007_11_26:human-group-form}

%
% debenham05:_multiag_system_manag_collab_emerg_proces

\begin{albResearchNote}{%
    \citet{debenham05:_multiag_system_manag_collab_emerg_proces}%
  }{%
    \glossary{multi-agent~system}%
  }
\item Defines \emph{emergent processes} as ``non-routine, collaborative
  business processes whose execution is guided by the knowledge that
  emerges during a process instance.''
\item Claims ``the goal of an emergent process instance may not be used
  as a focus for the process management system''.  Hence, the agent
  system manages the collaboration rather than the process
\item Observes that the integrity of information about the process
  decays with time.
\item Enumerates questions that an emergent process agent considers,
  including ``who to ask to assist'', ``who can I rely on'', ``who wroks
  well with who'', ``who do I want to build a relationship with''
\item Decomposes collaboration into the acts of \emph{joining}, and
  \emph{delegating} and claims information passing bears witness to
  these acts.
\end{albResearchNote}

%
%
% Bibliography
%

\bibliographystyle{plainnat}
\bibliography{example-bibliography}

% Local Variables:
% mode: latex
% TeX-master: "example-journal"
% ispell-local-dictionary: "british"
% End:
