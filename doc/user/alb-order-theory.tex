%%%
%%% AlbLaTeX/doc/user/alb-order-theory.tex
%%%
%%%     See copyright notice and license in text.
%%%
%%%   - Documentation for the 'alb-order-theory' LaTeX package.
%%%



\documentclass[11pt,a4paper,oneside]{alb-latex}



%
% URL Typesetting
%
% See: 'url.sty'.

\usepackage{url}


%
% The package being documented.

\usepackage{alb-order-theory}


%
% Sloppy Line Breaks
%
% Turn off careful line breaks and hyphenation.

\sloppy



\begin{document}



%%%
%%% TITLE
%%%

\albTitle{%
  Typesetting Order Theory Under \LaTeX{}%
}

\begin{albTitlePage}

  \albTitlePageSection{Author}

  Andrew Lincoln Burrow

  \albTitlePageSection{Abstract}

  The \texttt{AlbOrderTheory} package provides a single
  \texttt{alb-order-theory} \LaTeX{} package to provide markup for set
  and order theory.  It is designed to make the \LaTeX{} input more
  readable, to allow the actual symbols to be adjusted from a single
  file, to enable the typesetting of set and order theoretic
  constructions to be designed in the context of the entire suite of
  required markup commands, and to place the extended markup in the
  \albLogo{} namespace.  Therefore, the package provides a large
  collection of math commands to markup operators, relations, and
  functions.  The package is supported by an emacs lisp file customising
  \AUCTeX{}, which provides command name completion and argument
  prompting.


  \albTitlePageSection{Copyright}

  Copyright \copyright{} 1999--2006, 2013 Andrew Lincoln Burrow.\\
  This program may be distributed and/or modified under the conditions
  of the \LaTeX{} Project Public License, either version 1.3 of this
  license or (at your option) any later version.

  \medskip{}

  The latest version of this license is in
  \begin{quote}
    \url{http://www.latex-project.org/lppl.txt}
  \end{quote}
  and version 1.3 or later is part of all distributions of LaTeX version
  2005/12/01 or later.

  \medskip{}

  This work has the LPPL maintenance status `author-maintained'.

  \medskip{}

  This work consists of the files
  \begin{quote}
    \begin{flushleft}
      \url{alb-algorithms.sty}, \url{alb-avm.sty}, \url{alb-latex.cls},
      \url{alb-float-tools.sty}, \url{alb-graph-theory.sty},
      \url{alb-journal.cls}, \url{alb-order-theory.sty},
      \url{alb-proofs.sty}, \url{alb-theorems.sty},
      \url{alb-thesis.cls}, \url{alb-algorithms.tex}, \url{alb-avm.tex},
      \url{alb-latex.tex}, \url{alb-float-tools.tex},
      \url{alb-graph-theory.tex}, \url{alb-journal.tex},
      \url{alb-order-theory.tex}, \url{alb-proofs.tex},
      \url{alb-theorems.tex}, \url{alb-thesis.tex}.
      \url{alb-journal-glossary.ist}, \url{alb-journal-index.ist},
      \url{alb-thesis-glossary.ist}, and \url{alb-thesis-index.ist}.
    \end{flushleft}
  \end{quote}


  \albTitlePageSection{Version Information}

  \verb$Revision$\\
  \verb$Date$

\end{albTitlePage}



%%%
%%% INTRODUCTION
%%%

\section{Introduction}
\label{sec:alb-order-theory-documentation:intr}

The \texttt{alb-order-theory} \LaTeX{} package is designed to typeset
order theory.  In practice, this means that it supplies a large
collection of simple math commands.  The main benefit is consistency,
and the ability to easily change a decision about typesetting for all
documents.



%%%
%%% USING THE COMMANDS AND ENVIRONMENTS
%%%

\section{Using the Commands and Environments}
\label{sec:alb-order-theory-documentation:using-comm-envir}

The commands and environments of the \texttt{alb-order-theory} \LaTeX{}
package are divided into mathematical groups.  They are all used within
math mode.



\subsection{Logical Connectives}
\label{sec:alb-order-theory-documentation:logical-conn}

In lattice theory we use the vee and the wedge for join and meet.
Therefore, we lack an obvious symbol for the logical connectives.  The
current solution is just to use text labels.

\begin{quote}
  \begin{tabular}{p{0.5\textwidth}@{\qquad}p{0.3\textwidth}}
    % The negation.
    \verb$\albNot x$ & $\albNot x$ \\
    %
    % The and connective.
    \verb$x \albAnd y$ & $x \albAnd y$ \\
    %
    % The or connective.
    \verb$x \albOr y$ & $x \albOr y$
  \end{tabular}
\end{quote}



\subsection{Set Theory}
\label{sec:alb-order-theory-documentation:set-theory}

Commands accepting arguments and generating set theory mathematical
constructions.

\begin{quote}
  \begin{tabular}{p{0.5\textwidth}@{\qquad}p{0.3\textwidth}}
    %
    % The singleton set containing the element #1.
    \verb$\albSingleton{x}$ & $\albSingleton{x}$ \\
    %
    % The cardinality of the set #1.
    \verb$\albCardinality{X}$ & $\albCardinality{X}$ \\
    %
    % The tuple <#1>.
    \verb$\albTuple{x, y, z}$ & $\albTuple{x, y, z}$ \\
    %
    % The powerset of #1.
    \verb$\albPowerset{X}$ & $\albPowerset{X}$ \\
    %
    % The binary relation #1 over #2 x #3.
    \verb$\albBinaryRelation{R}{X}{Y}$ & $\albBinaryRelation{R}{X}{Y}$ \\
    %
    % The partial map #1 from #2 to #3.
    \verb$\albPartialMap{\alpha}{X}{Y}$ %
    & $\albPartialMap{\alpha}{X}{Y}$ \\
    %
    % The total map #1 from #2 to #3.
    \verb$\albTotalMap{\alpha}{X}{Y}$ %
    & $\albTotalMap{\alpha}{X}{Y}$ \\
    %
    % The domain of map #1.
    \verb$\albDomain{\alpha}$ & $\albDomain{\alpha}$ \\
    %
    % The range of map #1.
    \verb$\albRange{\alpha}$ & $\albRange{\alpha}$ \\
    %
    % The binary operator of composition: `f compose g'.
    \verb$f \albCompose g$ & $f \albCompose g$
  \end{tabular}
\end{quote}



\subsection{Standard Sets}
\label{sec:alb-order-theory-documentation:stand-sets}

Commands without arguments representing the standard sets of arithmetic.

\begin{quote}
  \begin{tabular}{p{0.5\textwidth}@{\qquad}p{0.3\textwidth}}
    %
    % The set of natural numbers.
    \verb$\albNaturals$ & $\albNaturals$ \\
    %
    % The set of real numbers.
    \verb$\albReals$ & $\albReals$
  \end{tabular}
\end{quote}



\subsection{Order Theory}
\label{sec:alb-order-theory-documentation:order-theory}

Commands accepting arguments and generating order theory mathematical
constructions.

\begin{quote}
  \begin{tabular}{p{0.5\textwidth}@{\qquad}p{0.3\textwidth}}
    %
    % The binary non-comparability relation: `x and y are incomparable'.
    \verb$x \albIncomparable y$ & $x \albIncomparable y$ \\
    %
    % The binary covering relation: `x is covered by y'.
    \verb$x \albCoveredBy y$ & $x \albCoveredBy y$ \\
    %
    % The binary covering relation: `x covers y'.
    \verb$x \albCovers y$ & $x \albCovers y$
  \end{tabular}

  \begin{tabular}{p{0.5\textwidth}@{\qquad}p{0.3\textwidth}}
    %
    % The equivalence class containing #1.
    \verb$\albEquivClass{x}$ & $\albEquivClass{x}$ \\
    %
    % The dual of #1.
    \verb$\albDual{P}$ & $\albDual{P}$
  \end{tabular}

  \begin{tabular}{p{0.5\textwidth}@{\qquad}p{0.3\textwidth}}
    %
    % The elements covering #2 in poset #1.
    \verb$\albOver{P}{x}$ & $\albOver{P}{x}$ \\
    %
    % The elements covered by #2 in poset #1.
    \verb$\albUnder{P}{x}$ & $\albUnder{P}{x}$ \\
    %
    % The maximum element of the poset #1.
    \verb$\albMax{A}$ & $\albMax{A}$ \\
    %
    % The minimum element of the poset #1.
    \verb$\albMin{A}$ & $\albMin{A}$ \\
    %
    % The set containing the maximal elements of the poset #1.
    \verb$\albHeads{P}$ & $\albHeads{P}$ \\
    %
    % The set containing the minimal elements of the poset #1.
    \verb$\albTails{P}$ & $\albTails{P}$
  \end{tabular}

  \begin{tabular}{p{0.5\textwidth}@{\qquad}p{0.3\textwidth}}
    %
    % The width of the poset #1.
    \verb$\albWidth{P}$ & $\albWidth{P}$ \\
    %
    % The height of the poset #1.
    \verb$\albHeight{P}$ & $\albHeight{P}$
  \end{tabular}

  \begin{tabular}{p{0.5\textwidth}@{\qquad}p{0.3\textwidth}}
    %
    % The binary join operator: `x join y'.
    \verb$x \albJoin y$ & $x \albJoin y$ \\
    %
    % The join of the elements in #1.
    \verb$\albJoinOf{A}$ & $\albJoinOf{A}$ \\
    %
    % The join in the ordered set #1 of the elements in #2.
    \verb$\albJoinInOf{P}{A}$ & $\albJoinInOf{P}{A}$ \\
    %
    % The binary meet operator: `x meet y'.
    \verb$x \albMeet y$ & $x \albMeet y$ \\
    %
    % The unary meet operator: `meet of S'.
    \verb$\albMeetOf{A}$ & $\albMeetOf{A}$ \\
    %
    % The meet in the ordered set #1 of the elements in #2.
    \verb$\albMeetInOf{P}{A}$ & $\albMeetInOf{P}{A}$
  \end{tabular}

  \begin{tabular}{p{0.5\textwidth}@{\qquad}p{0.3\textwidth}}
    %
    % The order embedding #1 from #2 to #3.
    \verb$\albOrderEmbedding{\alpha}{O}{P}$ &
    $\albOrderEmbedding{\alpha}{O}{P}$ \\
    %
    % The binary relation of order isomorphism: `P and Q are order-isomorphic.
    \verb$O \albOrderIsomorphic P$ & $O \albOrderIsomorphic P$
  \end{tabular}

  \begin{tabular}{p{0.5\textwidth}@{\qquad}p{0.3\textwidth}}
    %
    % The down-closure of the elements in #1.
    \verb$\albDown{A}$ & $\albDown{A}$ \\
    %
    % The up-closure of the elements in #1.
    \verb$\albUp{A}$ & $\albUp{A}$ \\
    %
    % The down-closure in the ordered set #1 of the elements in #2.
    \verb$\albDownInOf{P}{A}$ & $\albDownInOf{P}{A}$ \\
    %
    % The up-closure in the ordered set #1 of the elements in #2.
    \verb$\albUpInOf{P}{A}$ & $\albUpInOf{P}{A}$
  \end{tabular}

  \begin{tabular}{p{0.5\textwidth}@{\qquad}p{0.3\textwidth}}
    %
    % The lower-bounds of the set #1.
    \verb$\albLower{A}$ & $\albLower{A}$ \\
    %
    % The upper-bounds of the set #1.
    \verb$\albUpper{A}$ & $\albUpper{A}$ \\
    %
    % The lower-bounds in the ordered set #1 of the set #2.
    \verb$\albLowerInOf{P}{A}$ & $\albLowerInOf{P}{A}$ \\
    %
    % The upper-bounds in the ordered set #1 of the set #2.
    \verb$\albUpperInOf{P}{A}$ & $\albUpperInOf{P}{A}$
  \end{tabular}

  \begin{tabular}{p{0.5\textwidth}@{\qquad}p{0.3\textwidth}}
    %
    % The family of down-sets of #1.
    \verb$\albDownSets{P}$ & $\albDownSets{P}$ \\
    %
    % The rise of element #2 in poset #1.
    \verb$\albRise{P}{x}$ & $\albRise{P}{x}$ \\
    %
    % The fall of element #2 in poset #1.
    \verb$\albFall{P}{x}$ & $\albFall{P}{x}$ \\
    %
    % The Dedekind-MacNeille completion of #1.
    \verb$\albDM{P}$ & $\albDM{P}$ \\
    %
    % The semilattice of consistent down-sets of #1.
    \verb$\albCD{P}$ & $\albCD{P}$ \\
    %
    % The elements of poset #1 extending the consistent down-set #2.
    \verb$\albExtenders{P}{A}$ & $\albExtenders{P}{A}$ \\
    %
    % The elements of poset #1 retracting the consistent down-set #2.
    \verb$\albRetractors{P}{A}$ & $\albRetractors{P}{A}$
  \end{tabular}
\end{quote}



\subsection{Notation for the Growth of Functions}
\label{sec:alb-order-theory-documentation:notat-growth-funct}

Commands accepting arguments and generating expressions for the growth
of functions.

\begin{quote}
  \begin{tabular}{p{0.5\textwidth}@{\qquad}p{0.3\textwidth}}
    %
    % Asymptotic upper bound: big Oh notation.
    \verb$\albBigOh{f}$ & $\albBigOh{f}$ \\
    %
    % Asymptotic lower bound: big Omega notation.
    \verb$\albBigOmega{f}$ & $\albBigOmega{f}$ \\
    %
    % Asymptotic tight bound: big Theta notation.
    \verb$\albBigTheta{f}$ & $\albBigTheta{f}$
  \end{tabular}
\end{quote}



%%%
%%% AUCTEX CUSTOMISATIONS
%%%

\section{\AUCTeX\ Customisations}
\label{sec:alb-order-theory-documentation:auctex-cust}

Under \AUCTeX{} the file \texttt{alb-order-theory.el} is automatically
loaded whenever the \texttt{alb-order-theory} package is used.  The
customisation adds the math commands to \AUCTeX{}.  This provides the
simple prompting for all the supplied math commands.

In addition, \texttt{alb-order-theory.el} causes a special equation
number counter to be stored as a local variable.
\texttt{alb-LaTeX-equation-counter} counts numbers assigned to labels
for equations.  This ensures dirty numbers are not reissued, as
reissuing a number could make stale references hard to detect.



%%%
%%% MAKEFILE TARGETS
%%%

\section{Makefile Targets}
\label{sec:alb-order-theory-documentation:makef-targ}

The \texttt{AlbLaTeXDocumentTemplate} makefile provides a target to
relabel equations.  The \texttt{alb-relabel-eq} target edits the
\LaTeX{} source in an attempt to match equation labels to equation
numbers.  Labels of the form
\texttt{eq:}\albLtxPrm{identifier}\texttt{:}\albLtxPrm{number} are
processed.  \albLtxPrm{number} is rewritten to the last part of the
equation number.  This is helpful in proof reading.



\end{document}



%%% Local Variables:
%%% mode: latex
%%% TeX-master: t
%%% End:
